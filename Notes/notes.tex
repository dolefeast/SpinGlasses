\documentclass[a4paper]{article}


\usepackage[]{pgf}
\usepackage[utf8]{inputenc}
\usepackage[T1]{fontenc}
\usepackage{textcomp}
\usepackage{listings}
\usepackage[english]{babel}
\usepackage{amsmath, amssymb}


% figure support
\usepackage{import}
\usepackage{xifthen}
\pdfminorversion=7
\usepackage{pdfpages}
\usepackage{transparent}
\newcommand{\incfig}[1]{%
	\def\svgwidth{\columnwidth}
	\import{./figures/}{#1.pdf_tex}
}

\title{Spin glasses}
\author{Santiago Sanz Wuhl}

\pdfsuppresswarningpagegroup=1

\DeclareUnicodeCharacter{2212}{-}


\begin{document}
\maketitle

\section{Introduction}
A (classical) spin system with n particles is governed by the Hamiltonian 
\begin{align}
	H = \sum_{i=1}^{N} \sum_{1\le j < i} A_{ij} s_i s_j,
\end{align}
where $A_{ij}$ are the matrix elements of a 2-dimensional symmetric matrix, measuring the strength of the interactions between the particle at the site $i$ and the particle at the site $j$, and $s_i$ is the spin of the particle at the site $i$.
When only considering long-range interactions, 
\begin{align}
	A_{ij} = r_{ij}^{-\sigma}.
\end{align}

This interaction is however very computationaly expensive. These systems are usually approximated  (for example in the Ising model) via the nearest-neighbors approximation\footnote{This expression is valid only for 1-dimensional setups. The expression for higher dimensions is similar.}, i.e. 
\begin{align}
	A_{ij}=
	\begin{cases}
		1 & j = i \pm 1\\
		0 & \text{else}
	\end{cases}.
\end{align}

To avoid the computational expense of calculating the $\frac{N(N-1)}{2}$ interactions per iteration we state the following approximation: the interaction strength of any two particles is always 1, independently of the  distance separating them, but not every particle interacts with every particle. We distribute a fixed $N_l$ amount of interactions  among the particles, and the probability that two particles are interacting is governed by $r_{ij}^{-\sigma}$.

\section{Methods}
\subsection{Random number generation}

The core of these calculations is the Walker's Alias random number generating algorithm. This algorithm rolls an unfair $m$ sided die with complexity $\mathcal{O}(m)$.

\subsection{Two different algorithms}

We discussed two different algorithms to generate the matrix $A_{ij}$ (which is not anymore a matrix, it is only the set of 1 valued elements)

\paragraph{Set algorithm}
This algorithm draws random bonds between any two particles and when the number of bonds reaches $N_l$, it stops.
Each draw of a bond rolls two dice: The first die draws a uniformly distributed integer $n$ from 1 to N and chooses the starting particle. The second die draws a random integer from 1 to $\left\lfloor \frac{N}{2} \right\rfloor$ where the probability of drawing a number number $m \in [1, \left\lfloor \frac{N}{2} \right\rfloor]$ is 
\begin{align}
	P_m = m^{-\sigma}.
\end{align}
This, again only draws "forward" bonds.
The drawn bond is then $\{n, \text{mod}\left(n+m,  N  \right) \}$. This avoids having to generate the Alias Table at each iteration

The algorithm goes as follows
\begin{lstlisting}
Parameters: N, N_l, sigma

# First particle choice AliasTable
uniformAliasTable = AliasTable(ones(N))	
# Second particle choice AliasTable
distanceAliasTable = AliasTable( 1 ./ 1:div(N, 2)  .^ sigma )
\end{lstlisting}

\end{document}
